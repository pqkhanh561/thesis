\documentclass[12pt,a4paper]{article}
\usepackage{amsmath,amsthm,amssymb}
\usepackage[utf8]{vietnam}
\usepackage{blindtext}
\usepackage{bbm}
\usepackage{listings}
\renewcommand{\baselinestretch}{1.5}
\title{Section 2: Multi-armed bandits}
\author{Phan Quang Khánh - 1611125}
\begin{document}
\maketitle
\section{A k-armed Bandit Problem}
Tưởng tượng đây là bài toán xác suất khi 
cho con bạc tuột chơi trò slot machine. Tức mỗi lần con bạch tuột sẽ lựa chọn tua để sử dụng và mỗi lần chơi sẽ có phần thưởng được nhận

\section{Action-value Method}
Xây dựng hàm value để ước lượng được giá trị mỗi lần tiếp theo chọn hành động để thực hiện. Ví dụ nếu con bạch tuột chọn máy 1 để nhấn thì phải có hàm để xác định xem lần nó nhấn có thực sự là tốt không?\\
Bình thường hàm để tính được định nghĩa đơn giản như sau:
\begin{align*}
Q_t(A) &= \dfrac{\text{Tổng của số lần Reward của hành động thực hiện a từ trước khi đến t}}{\text{Tổng số lần chọn hành động a từ trước khi đến t}}\\
&= \dfrac{\sum_{i = 1}^{t-1} R_i.\mathbbm{1_{A_i = a}}}{\sum_{i = 1}^{t-1} \mathbbm{1_{A_i = a}}}
\end{align*}
Với $\mathbbm{1}$ là hàm đặc trưng cho việc chọn hành động hay không.
\section{The 10-armed testbed}

\end{document}