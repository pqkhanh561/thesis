Khóa luận này hướng đến việc áp dụng phương pháp học tăng cường bằng thuật toán Q-learning cùng với các tùy chỉnh để chiến thắng trò chơi World's Hardest Game cho mục đích chung giải quyết bài toán tìm đường đi trong môi trường động với số bước thực hiện tối ưu. Nhóm tác giả đã đưa ra quá trình cài đặt môi trường khi đối mặt với nhiều ngôn ngữ cùng với các kinh nghiệm khi thực hiện huấn luyện trên chúng. Ngoài ra đưa ra giả thiết về trạng thái của môi trường có thể ảnh hướng đến kết quả. Các phương pháp thử nghiệm và kết quả được trình bày trong chương \ref{Chap3} và chương \ref{Chap4}. Phần thưởng là yếu tố then chốt trong việc cải thiện cách học của mô hình trong phương pháp học tăng cường. Nhiều sự cải thiện qua những thay đổi trong sự thúc đẩy và hình phạt được nhóm thực hiện trong luận văn. \\
\\
Mô hình chính mà nhóm nghiên cứu là sử dụng mạng nơ-ron để ước lượng giá trị Q sau khi thực hiện phân tích đặc trưng của môi trường. Ngoài ra, hai thành phần giúp cải thiện mô hình được sử dụng là mạng mục tiêu để tránh sự dao động quá nhiều và lịch sử kinh nghiệm giúp hàm mất mát không hội tụ nhanh trong quá trình tìm chính sách. Kết quả của mô hình không được như mong đợi khi đạt tỉ lệ thắng 55,8\% chứng tỏ rằng còn rất nhiều cải tiến có thể thực hiện với môi trường và mô hình hiện tại.    